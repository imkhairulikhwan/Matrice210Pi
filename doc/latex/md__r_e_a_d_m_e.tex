This code uses \href{https://github.com/dji-sdk/Onboard-SDK/}{\tt D\+JI Onboard S\+DK} to communicate with D\+JI flight controllers.

\subsection*{Prerequisites}

Code compiled with \href{https://cmake.org/}{\tt C\+Make} 3.\+7.\+2 on a \href{https://www.raspberrypi.org/products/raspberry-pi-zero-w/}{\tt Raspberry Pi Zero W} with \href{https://www.raspberrypi.org/downloads/raspbian/}{\tt Raspbian Stretch Lite} (June 2018 -\/ Kernel version\+:4.\+14).

\subsubsection*{U\+A\+RT}

\href{https://github.com/dji-sdk/Onboard-SDK/}{\tt D\+JI Onboard S\+DK} uses serial U\+A\+RT driver to communicate with D\+JI flight controllers. The U\+A\+RT transmit and receive pins are on G\+P\+IO 14 and G\+P\+IO 15 respectively, which are pins 8 and 10 on the G\+P\+IO header.

In a default install of Raspbian on a Raspberry, the primary U\+A\+RT {\ttfamily /dev/serial0} is assigned to the Linux console. To stop this behaviour, the serial console setting needs to be removed from command line. This can be done using the https\+://www.raspberrypi.\+org/documentation/configuration/raspi-\/config.md \char`\"{}raspi-\/config\char`\"{} \+:


\begin{DoxyCode}
sudo raspi-config
\end{DoxyCode}
 Select option 5, {\bfseries Interfacing options}, then option {\bfseries Serial}, then {\bfseries Disable serial login shell} (disable linux\textquotesingle{}s use of console uart) and {\bfseries Enable serial interface}. Exit raspi-\/config and reboot

\subsubsection*{U\+A\+RT by U\+SB}

The \href{https://www.st.com/en/evaluation-tools/32f429idiscovery.html}{\tt 32\+F429\+I\+D\+I\+S\+C\+O\+V\+E\+RY} card used by the embedded sensor transmits values by U\+A\+RT. It is connected to a U\+S\+B-\/\+T\+TL adaptator and uses U\+SB User port {\ttfamily /dev/tty\+U\+S\+B0} on the Pi Zero W.

\subsection*{Install}

Clone the \href{https://github.com/dji-sdk/Onboard-SDK/}{\tt D\+JI Onboard S\+DK} repository and configure {\ttfamily O\+N\+B\+O\+A\+R\+D\+S\+D\+K\+\_\+\+S\+O\+U\+R\+CE} in \href{CMakeLists.txt}{\tt C\+Make\+List.\+txt} depending on your repository location.

In the {\ttfamily Matrice210\+Pi} root directory, run the following commands to build the app\+: 
\begin{DoxyCode}
mkdir build && cd build
cmake ..
make
cd bin
\end{DoxyCode}


You have to copy the {\ttfamily User\+Config.\+txt} file provided by D\+JI in the {\ttfamily bin} directory and fill it in with your configuration informations. You must register as a developer with D\+JI and create an O\+S\+DK application ID and Key pair, see \href{https://developer.dji.com/onboard-sdk/documentation/development-workflow/environment-setup.html#onboard-sdk-application-registration}{\tt here}.

Be sure Onboard S\+DK is enabled on the aircraft and the baudrate used in {\ttfamily User\+Config.\+txt} file is the same as defined with D\+JI Assistant 2, more informations \href{https://developer.dji.com/onboard-sdk/documentation/development-workflow/environment-setup.html}{\tt here}.

More informations can be found in the \href{https://developer.dji.com/onboard-sdk/documentation/quick-start/quick-start.html}{\tt Quick start Guide} or in the \href{https://developer.dji.com/onboard-sdk/documentation/introduction/homepage.html}{\tt D\+JI documentation}.

You can then launch program in {\ttfamily bin} directory with 
\begin{DoxyCode}
sudo ./matrice210 1
\end{DoxyCode}


If the program is launched with {\ttfamily sudo ./matrice210 0}, the console interface is not displayed.

\subsection*{Result}

The following interface is shown in the program console when console is enabled.



Due to the watchdog, if the program is launched after the Android Application on the mobile device, data may be received before full program initialization.



If console is disabled, {\ttfamily Available commands} are not displayed.

\subsection*{Linux service}

The \href{Linux/runMatrice210.sh}{\tt run\+Matrice210.\+sh} script can be automatically launched from a service on Pi start-\/up if the \href{Linux/matrice210.service}{\tt matrice210.\+service} is added in {\ttfamily /etc/systemd/system}. Linux service can then be \href{Linux/stopMatrice210.sh}{\tt stopped} and \href{Linux/startMatrice210.sh}{\tt restarted} with dedicated script files

\subsection*{Log}

The \href{Linux/runMatrice210.sh}{\tt run\+Matrice210.\+sh} saves console output in {\ttfamily build/bin/log/} directory. Logs are formatting as follow \+: {\ttfamily log\mbox{[}index\mbox{]}-\/\mbox{[}yyyy\mbox{]}\mbox{[}mm\mbox{]}\mbox{[}dd\mbox{]}-\/\mbox{[}hh\mbox{]}\mbox{[}mm\mbox{]}\mbox{[}ss\mbox{]}.log}. G\+MT Date/\+Time is used, {\ttfamily index} is incremented to have numbered log and last log starts with \+\_\+ char.

The current log file can be read in real time with command {\ttfamily tail -\/f \+\_\+$\ast$.log}.

\subsection*{Usage}

For full compatibility, use this code with the \href{https://github.com/jonathanmichel/Matrice210Android}{\tt Matrice210\+Android\+App} on an Android device connected to aircraft remote controller and the \href{https://github.com/jonathanmichel/Matrice210Stm32}{\tt Matrice210\+Stm32} code running on a \href{https://www.st.com/en/evaluation-tools/32f429idiscovery.html}{\tt S\+T\+M32\+F429\+I\+D\+I\+S\+C\+O\+V\+E\+RY board}.

\subsection*{Authors}


\begin{DoxyItemize}
\item {\bfseries Jonathan Michel} -\/ {\itshape Initial work} -\/ \href{https://github.com/jonathanmichel}{\tt jonathanmichel} 
\end{DoxyItemize}